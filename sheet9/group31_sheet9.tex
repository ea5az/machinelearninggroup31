\documentclass{article}
\usepackage{tikz}
\usepackage{mathtools}

\newcommand\then{\rightarrow}
\newcommand\liff{\leftrightarrow}
\newcommand\lxor{\oplus}
\author{Sandra Kohl, Jan Hendrik Kirchner, Max Bernhard Ilsen}

\begin{document}
\section{Exercise \textit{(k-nearest Neighbor (4p))}}
Mrs.A who studies Cognitive Science is looking for a T-shirt for her
boyfriend, whose weight is about 80 kg and 177 cm tall. Please help her
to find the right T-shirt size using simple k-Nearest Neighbor and
Euclidean distance. To be certain, pick k=1,3 and 5.\\
\\
Distances of $x=(177,80)$ to each other data point:
\begin{align*}
    d(x,x_1) &= \sqrt{(177-188)^2+(80-100)^2} &= \sqrt{521} &= 22.8254\\
    d(x,x_2) &= \sqrt{(177-178)^2+(80-108)^2} &= \sqrt{785} &= 28.0178\\
    d(x,x_3) &= \sqrt{(177-170)^2+(80-50)^2} &= \sqrt{949} &= 30.8058\\
    d(x,x_4) &= \sqrt{(177-180)^2+(80-86)^2} &= 3\sqrt{5} &= 6.7082\\
    d(x,x_5) &= \sqrt{(177-193)^2+(80-70)^2} &= 2\sqrt{89} &= 18.868\\
    d(x,x_6) &= \sqrt{(177-182)^2+(80-61)^2} &= \sqrt{386} &= 19.6469\\
    d(x,x_7) &= \sqrt{(177-187)^2+(80-70)^2} &= 10\sqrt{2} &= 14.1421\\
    d(x,x_8) &= \sqrt{(177-173)^2+(80-93)^2} &= \sqrt{185} &= 13.6015\\
    d(x,x_9) &= \sqrt{(177-172)^2+(80-80)^2} &&= 5\\
    d(x,x_{10}) &= \sqrt{(177-185)^2+(80-92)^2} &= 4\sqrt{13} &= 14.4222\\
    d(x,x_{11}) &= \sqrt{(177-174)^2+(80-80)^2} &&= 3\\
    d(x,x_{12}) &= \sqrt{(177-174)^2+(80-70)^2} &= \sqrt{109} &= 10.4403\\
\end{align*}
Since we are dealing with discrete valued output, we take the target value that
occurs most often among the k nearest neighbors as the target value for $x$.\\
\begin{itemize}
    \item
        $k=1$-nearest neighbors:\\
        $x_{11}=(174,80), t_{11}=XL$\\
        Choose $t = XL$.\\
    \item
        $k=3$-nearest neighbors:\\
        $x_{11}=(174,80), t_{11}=XL$\\
        $x_9=(172,80), t_9=XL$\\
        $x_4=(180,86), t_4=M/L$\\
        Choose $t = XL$.\\
    \item
        $k=5$-nearest neighbors:\\
        $x_{11}=(174,80), t_{11}=XL$\\
        $x_9=(172,80), t_9=XL$\\
        $x_4=(180,86), t_4=M/L$\\
        $x_{12}=(174,70), t_{12}=M/L$\\
        $x_8=(173,93),t_8=XL$\\
        Choose $t = XL$.\\
\end{itemize}

\section{Exercise \textit{(RBF (8p))}}
\begin{enumerate}
    \item Discuss RBF network and MLP in different aspects e.g. input and output
        dimension, extrapolation, lesion tolerance and advantages of each
        network.\\
        \\
    \item The training of RBF network concerns three parts. The first step is to
        find suitable centers or input weights, $\xi$. Explain in detail how to
        find these input weights.\\
        \\
    \item  Write down another basis function which has the property $\Phi(r)
        \then 0$ as $|r| \then \infty$ and one example a of basis function
        which has property: $\Phi(r) \then \infty$ as $|r| \then
        \infty$.\\
        \\
\end{enumerate}

\section{Exercise \textit{(SOM (8p))}}
\begin{enumerate}
    \item Explain\\
        \begin{enumerate}
            \item the meaning of topology preservation:
            \item the properties of the topology function:
            \item measuring similarity in SOM:
        \end{enumerate}
    \item How to avoid that the later training phases forcefully pull the entire
        map towards a new pattern?\\
        \\
    \item  Briefly discuss at least three applications of SOM in different aspects.\\
        \\
\end{enumerate}
\end{document}
